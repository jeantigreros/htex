\documentclass{article}
\author{Alexis Tigreros}
\date{10 de marzo de 2025}

\begin{document}
\title{Red 3: El colorante cancerígeno en la comida}
\maketitle
Los colorantes en alimentos, medicamentos o cosméticos se utilizan para mejorar su apariencia y hacerlos más atractivos para los consumidores. En la industria alimentaria, por ejemplo, los colores pueden reforzar la percepción del sabor y compensar la pérdida de color durante el procesamiento. El uso de estos aditivos está regulado por entidades como la FDA en Estados Unidos o el Invima en Colombia para garantizar que sean seguros y no representen riesgos
para la salud del consumidor. ¿Qué significa esta prohibición? - ¿Qué es el Red 3 y para qué se utiliza? - ¿Cuál es la situación en Colombia? - ¿Existen alternativas en el mercado?
\section{¿Qué es el Red 3 y para qué se utiliza?}
El Red 3 (Rojo No 3), también conocido como Eritrosina, es un colorante sintético de la familia de
los xantenos, cuya estructura química se basa en un anillo de fluorona con sustituyentes de yodo, Figura 1. Su nombre químico es tetrayodofluoresceína sódica (C₂₀H₆I₄Na₂O₅), y se caracteriza por su intenso color rojo. Es una sal altamente soluble en agua, utilizada en alimentos, medicamentos y cosméticos.
 
El Red 3 se encuentra incluido en la Norma General del Codex para los Aditivos Alimentarios (GSFA,
Codex STAN 192-1995), con el número SIN.127, para uso como colorante en dosis máximas que oscilan entre 30 y 200 mg/kg de producto, en 7 categorías de alimentos y bebidas.
\section{¿Qué significa la prohibición del Red 3?}
Según la nueva normativa de la FDA, los fabricantes que utilicen Red 3 en alimentos, medicamentos ingeridos o cosméticos tendrán plazo hasta enero de 2027 para reformular sus productos. La prohibición también se aplicará a los productos importados. Esto implica la reformulación de productos que contienen este colorante y se pretenden exportar a los Estados Unidos
Esta decisión sigue la recomendación de la cláusula Delaney de 1958, que establece que no se puede aprobar el uso de ningún aditivo alimentario sobre el cual haya evidencia de que causa cáncer en humanos o animales.
\section{En conclusión}
Es necesario revisar las alternativas que existen en el mercado y aunar esfuerzos entre la academia, el estado y los industriales para desarrollar los aditivos que se necesitan en el corto plazo para reducir los riesgos en el consumidor y evitar la posible afectación en las exportaciones a Estados Unidos, nuestro principal aliado comercial.
\section{¿Cómo se pueden articular?}
Academia: utilizar la infraestructura y el valioso talento humano del país para proponer alternativas de producción de estos importantes aditivos, ya sea utilizando la química sintética [6], la biotecnología [7] o una combinación entre ellas.
\textbf{Estado}
: crear convocatorias/concursos para incentivar la producción de conocimiento aplicable a la producción de estos aditivos.
\textbf{Industriales}
: apoyar la financiación de proyectos que tengan como objetivo el desarrollo de procesos viables para la manufactura de estos colorantes.
Finalmente, desde este espacio se invita al 
\textbf{Invima}
, al 
\textbf{Ministerio de Salud }
y la 
\textbf{Protección Social}
 y a la 
\textbf{Academia}
 a revisar la dosis máxima permitida, que es 100mg/kg mayor a la que recomienda el Codex Alimentarius. Por otro lado, aunar esfuerzos para hacer el seguimiento detallado a este aditivo ya que las evidencias son muy claras sobre sus potenciales efectos adversos para los consumidores

\end{document}
